%%This is a very basic article template.
%%There is just one section and two subsections.
\documentclass[10pt,oneside,a4paper,fleqn]{article}
\usepackage[utf8]{inputenc}
\usepackage[ngerman]{babel,varioref}
\usepackage{amssymb}
\usepackage{a4wide}

\usepackage{pgfplots}
\usepackage{tikz}
%\usepackage{tikzpicture}

\title{Laborjournal NAP}
\author{Damian Köppel, Gian Claudio Köppel}
\begin{document}

\maketitle

\section{Ionentauscher}

\subsection{Spühlen des Ionentauschers}

Zu Beginn lassen wir die noch in den Tauschern vorhandene Flüssigkeit ab.
Vermutlich handelt es sich dabei um deionisertes Wasser, aber wir können uns
nicht sicher sein. Dabei achten wir darauf, dass die Tauscher nie an die Luft
kommen. Luftblasen die allenfalls im Tauscher eingeschlossen werden,
könnten den gesamten Ionentauscherprozess beeinflussen. Danach spülen wir die
Tauscher mit deionisiertem Wasser, bis das Wasser das aus den Tauschern fliesst
den selben pH-Wert wie das Deioniserte Wasser hat, das in den Tauscher gegeben
wird. Damit stellen wir sicher, dass sich keine unbekannte Stoffe mehr im
Tauscher befinden. Der pH-Wert wird mittels Messtreifen besstimmt, in unserem
Fall beträgt der pH-Wert des Wassers 6. Nach der Spühlung hat die Flüssigkeit
aus beiden Tauschern den pH-Wert 6.

\subsection{Kalibration der pH-Messgeräten}
Wir kalibirieren das pH-Meter vom Typ \textit{Methrom 827 PH Lab} mit
einer dreipunkte Kalibration. Wir verwenden dazu Pufferlösungen
mit den pH-Werten 4, 7 und 10. Vor der Kalibration und bei jedem Wechsel der
Puffer spülen wir die Glaselektrode mit deionisiertem Wasser, um
Verunreinigungen, die das Messergebnis beeinflussen, möglichst klein zu halten.
Die bei der Kalibration gemessenen Werte können den Tabellen \ref{tab:ersteKal}
und \ref{tab:zweiteKal} entnommen werden.
Die zweite Kalibration wurde nötig, da die erste Kontrollmessung mit der pH 7
Pufferlösung ausserhalb der geforderten Genauigkeit von $\pm 0.1$ lag.

\begin{table}[!h]
\caption{erste Kablibtration}
\label{tab:ersteKal}
\centering
	\begin{tabular}{|l|l|l|}
		\hline
		Puffer & pH-Wert & Elketrodenspanung $\left[mV\right]$\\
		\hline
		4 & 3.86 &\\
		\hline
		7 &  & -2.3\\
		\hline
		10 &  & -169.4\\
		\hline
	\end{tabular}
\end{table}

\begin{table}[!h]
\caption{zweite Kalibtraion}
\label{tab:zweiteKal}
\centering
	\begin{tabular}{|l|l|l|}
		\hline
		Puffer & pH-Wert & Elketrodenspanung $\left[mV\right]$\\
		\hline
		4 &  & 172.8\\
		\hline
		7 & 7.09& -2.7635\\
		\hline
		10 &  & -168.1\\
		\hline
	\end{tabular}
\end{table}

\newpage

\begin{figure}[!h]
	\centering
	\begin{tikzpicture}[scale=0.5]
	%Achsen
	\draw[->] (-0.2,0) -- (11,0) node[right] {pH};
	\draw[->] (0,-6) -- (0,10.5) node[above] {U [mV]};
	
	%Punkte Messung
	\node [fill=red,inner sep=2pt] at (4,4.32) {}; %,label=-45:$F$
	\node [fill=red,inner sep=2pt] at (10,-4.2025) {};
	
	%Punkt Kontrolle
	\node [fill=green,inner sep=2pt] at (7.09,-0.06908) {};
	
	%Linie
	\draw [color=blue] (0,10) -- (11,-5.622);
	
	%x-Achsenbeschriftung
	\draw (1,0.2) -- (1,-0.2) node[anchor=north] {$1$};
	\draw (2,0.2) -- (2,-0.2) node[anchor=north] {$2$};
	\draw (3,0.2) -- (3,-0.2) node[anchor=north] {$3$};
	\draw (4,0.2) -- (4,-0.2) node[anchor=north] {$4$};
	\draw (5,0.2) -- (5,-0.2) node[anchor=north] {$5$};
	\draw (6,0.2) -- (6,-0.2) node[anchor=north] {$6$};
	\draw (7,0.2) -- (7,-0.2) node[anchor=north] {$7$};
	\draw (8,0.2) -- (8,-0.2) node[anchor=north] {$8$};
	\draw (9,0.2) -- (9,-0.2) node[anchor=north] {$9$};
	\draw (10,0.2) -- (10,-0.2) node[anchor=north] {$10$};
	%y-Achsenbeschriftung
	\draw (0.2,-5) -- (-0.2,-5) node[anchor=east] {$-200$};
	\draw (0.2,-2.5) -- (-0.2,-2.5) node[anchor=east] {$-100$};
	\draw (0.2,0) -- (-0.2,0) node[anchor=east] {$0$};
	\draw (0.2,2.5) -- (-0.2,2.5) node[anchor=east] {$100$};
	\draw (0.2,5) -- (-0.2,5) node[anchor=east] {$200$};
	\draw (0.2,7.5) -- (-0.2,7.5) node[anchor=east] {$300$};
	\draw (0.2,10) -- (-0.2,10) node[anchor=east] {$400$};
	
	%Legende
	\node [fill=red,inner sep=2pt,label=0:$Messwert$] at (12,5) {};
	\node [fill=green,inner sep=2pt,label=0:$Kontrollwert$] at (12,4) {};
	\draw [color=blue] (11.8,3) -- (12.2,3);
	\node [label=0:$Lin Reg$] at (12,3) {};
\end{tikzpicture}
	\caption{Kalibration der pH-Messung}
	\label{fig:pHKalibration}
\end{figure}

Durch die Lineare Regression konnten wir eine Gleichung für
$Leitf"ahigkeit=f(pH\ Wert)$ herleiten:
$$Leitf"ahigkeit = -56.81*(pH\ Wert)+400.06$$\\
Beziehungsweise:
$$pH\ Wert =  -\frac{Leitf"ahigkeit-400.06}{56.81}$$
Bei einem pH Wert von 4.8 ergibt das also eine Leitfähigkeit von 127.34.

\subsection{Regeneration der Ionentauscher}
Um die volle Funktion der Ionentauschern garantieren zu können, müssen diese
regeneriert werden. Den Kationentauscher wird mit $80 ml$ 8\%
$HCl$-Lösung regeneriert. Der Anionentauscher wird mit $80 ml$  0.5 M
$NaOH$-Lösung regeneriert. Die Regenerationslösungen lassen wir mit einer
Geschwindigkeit von ungefähr 2 Tropfen pro Sekunden durch die Tauscher.

Danach spühlen wir die Tauscher wieder mit deionierstem Wasser.
Um den Spühlvorgang überwachen zu können, überprüfen wir laufend den pH-Wert
mittels Messstreifen. Ein Ionentauscher gilt als gespühlt wenn der pH-Wert der
Spühlflüssigkeit vor dem Spühlen und danach übereinstimmen. Die Messewerte können
 der Tabelle \ref{spuehl} entnommen werden. Die Spühlflüssigkeit weist
einen pH-Wert von 6 auf. Auch bei der Spühlung verwenden wir eine
Ablassgeschwindigkeit von ca. 2 Tropfen pro Sekunde.

\begin{table}[h]
\caption{Messungen zur Überwachung des Spühlvorgangs}
\label{spuehl}
\centering
	\begin{tabular}{|l|l|l|}
	\hline
	Spühlwasser $\left[ml\right]$ & Anionentauscher $\left[pH\right]$ &
	Kathionentauscher $\left[pH\right]$\\
	\hline
	0 & 12 & 1 \\
	\hline
	170 & 11 & 1.5 \\
	\hline
	270 & 8 & 3 \\
	\hline
	300 & 7 & 3\\
	\hline
	350 & 6 & 4\\
	\hline
	400 & & 6 \\ 
	\hline
	\end{tabular}
\end{table}

\subsection{Kationentausch}
\label{sec:kat}
Durch den Kationentauscher lassen wir $2 \cdot 80 ml$ $CaCl_2$ Prüflösung mit
einer Tropfgeschwindikgeit von 2 Tropfen pro Sekunde laufen und verwerfen diese.
Danach lassen wir $3 \cdot 80 ml$ Prüflösung durch den Tauscher mit der selben
Geschwindigkeit wie die verworfene Lösung laufen. Diese zweite Tranche
der teilentioniserten Lösung fangen wir in einem sauberen $500 ml$ Gefäss zur
weiterverwendung auf. Gemäss Versuchsanleitung wäre hier ein $300 ml$ Gefäss
verlangt, allerdings war kein sauberes solches Vorhanden. $20 ml$ der erhalten
Lösung stellen wir für Messungen beiseite.

\subsection{Anionentausch}
Die beim \ref{sec:kat} Kationentausch erhaltene $Ca^+$-Freie Lösung wird jetzt
durch den Anionentauscher geschickt. Dies machen wir wieder in zwei
Tranchen, von denen wir die Erste verwerfen. Entgegen der
Versuchsanleitung bestand unsere erste Tranche aus $140 ml$ anstelle von $160ml$ da wir $20 ml $ für Messungen
beiseite Gestellt haben. Die Zweite Tranche von $80 ml$ vollentsalzter
Probelösung fangen wir in einem sauberen $100ml$
Becherglas auf.

\subsection{Messergebnisse}
Folgene Ergebinsse ergaben sich bei der Auswertung der Versuche.

\subsubsection{pH-Messung}
Die pH-Werte messen wir mit einem Glaselektrodenmessgerät vom Typ
\textit{Metrohm 827 PH Lab}. Um die für die Messung notwendige Leitfähigkeit zu
erreichen haben wir die vollentsalzte Lösung mit $NaCl$ aufgesalzen.

\begin{table}[h]
\caption{pH-Werte}
\label{tab:pHMess}
\centering
	\begin{tabular}{|l|l|l|}
		\hline
		Lösung & pH-Wert & Elektrodenspannung $\left[mV\right]$ \\
		\hline
		$CaCl_{2}$Probelösung & 5.82 & -65 \\
		\hline
		$Ca^{2+}$ freie Lösung & 1.30 & 326 \\
		\hline
		Vollentsalzte Lösung & 8.54 & -85\\
		\hline
	\end{tabular}
\end{table}

\subsubsection{Leitfähigkeit}
Die Leitfähigkeit der Lösungen messen wir mit einem Glaselektrodenmessgerärt vom
Typ \textit{Metrohm 712 Conductometer}.

\begin{table}[h]
\caption{Leitfähigkeit}
\label{tab:leitfaehigkeit}
\centering
	\begin{tabular}{|l|l|}
		\hline
		Lösung & Leitfähigkeit $\left[\frac{\mu s}{cm}\right]$\\
		\hline
		$CaCl_{2}$Probelösung & 4.874 \\
		\hline
		$Ca^{2+}$ freie Lösung & 17.29\\
		\hline
		Vollentsalzte Lösung & 18.03 \\
		\hline
	\end{tabular}
\end{table}

\subsubsection{Qualitativer $Ca_{2+}$ und $Cl^{-}$Ionen}
Die Ionen können mit folgenden Methoden nachgewiesen werden.
\begin{itemize}
  \item[$Ca^{2+}$] Zwei Tropfen Probelösung mit einem Tropfen $H_2SO_4$
  versetzen. Nach 1-3 Minuten sind unter dem Mirkroskop Gipsnadeln sichtbar.
  \item[$Cl^-$] Ein Tropfen Probelösung mti eingen Tropfen $AgNO_{3}$-Lösung
  versetzen. Es wird ein weisser, flockiger $AgCl$-Niederschlag sichbar.
\end{itemize}

\begin{table}[h]
\caption{Qualitativer Ionennachweis}
\label{tab:ionenNachweis}
\centering
	\begin{tabular}{|l|c|c|}
		\hline
		Lösung & $Ca^{2+}$ vorhanden & $Cl^-$ vorhanden\\
		\hline
		$CaCl_{2}$Probelösung & + & + \\
		\hline
		$Ca^{2+}$ freie Lösung & - & + \\
		\hline
		Vollentsalzte Lösung & - & -\\
		\hline
	\end{tabular}
\end{table}

\subsection{Menge gelöstes $CaCl_2$ in Probelösung}
\label{mengecacl}
pH-Wert Probelösung nach Kationentausch: 1.3 \newline
Stoffmege in der Lösung:
$c\left[H_3O^+\right]10^{-pH}=10^{-1.3}\frac{mol}{l}$\newline
\textbf{Molmassen:}
\begin{itemize}
	\item $M\left( Ca \right)=40.078 u$
	\item $M\left( CL \right)=35.45 u$
	\item $M\left(CaCl_2\right)= 40.078+2*35.45=110.978\frac{g}{mol}$
\end{itemize}

$$Ca^{2+} \textrm{Ionen werden durch } 2 \cdot H_3O \textrm{ Ionen
ersetzt.}$$
$$\Rightarrow c\left[Ca^{2+}\right] = 0.5 \cdot c\left[H_3O^+\right] \Rightarrow
c\left[Ca^{2+}\right] = 0.5 * 10^{-1.3}\frac{mol}{l}$$
$$1\textrm{ mol }Ca^{2+} \textrm{ reagiert zu } 1\textrm{ mol }CaCl_2$$
$$\Rightarrow Ca^{2+}+ 2Cl^- \Leftrightarrow CaCl_2 \Rightarrow
c\left[CaCl_2\right]= c\left[Ca^{2+}\right] = 0.5 \cdot 10^{-1.3}\frac{mol}{l}$$
$$\textrm{Konzentration } CaCl_2=0.5 \cdot
10^{-1.3}\frac{mol}{l}\cdot 110.978 \frac{g}{mol}= 2.78 \frac{g}{l}$$

\subsection{Interpretation}
Wie wir in der Tabelle \ref{tab:ionenNachweis} aufgezeigt haben, konnten wir
beim vollentsalzten Wasser keine $Ca^{2+}$ oder $Cl^-$ Ionen nachweisen. Damit
können wir davon ausgehen das die Vollentsalzung funktioniert hat. Da wir keine
$Ca^{2+}$ Ionen nachweisen konnten, können wir davon ausgehen, dass das Wasser
voll enthärtet wurde.\\
Die bei \ref{mengecacl} berechnete Masse von $2.78\frac{g}{l}$ $CaCL_2$ Ionen
erscheint uns, basierend auf unseren Chemiekenntnissen, einigermassen
realistisch.\\
Bei keinem der Messwerte sind uns bemerkenswerte Abweichungen von den
Theoriewerten aufgefallen.

\end{document}