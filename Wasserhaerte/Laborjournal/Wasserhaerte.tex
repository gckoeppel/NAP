%%This is a very basic article template.
%%There is just one section and two subsections.
\documentclass[10pt,oneside,a4paper,fleqn]{article}
\usepackage[utf8]{inputenc}
\usepackage[ngerman]{babel,varioref}
\usepackage{amssymb}
\usepackage{a4wide}

\usepackage{multicol}
\usepackage{pgfplots}
\usepackage{tikz}
\usepackage{listings}
\usepackage[numbered]{mcode}
%\usepackage{tikzpicture}

\title{Laborjournal NAP}
\author{Damian Köppel, Gian Claudio Köppel, Gruppe 1402}
\begin{document}

\maketitle

\section{Härtebestimmung}
\subsection{Probewasserentnahme}
Als erstes spühlen wir die Wasserleitungen, damit wir frisches Wasser entnehmen
konnten, da bei abgestandenem Wasser andere Werte gemessen werden könnten. Dazu
öffnen wir die wir die beiden Handventile und lassen mithilfe der Wasseruhr
ca 20 Liter in den Abguss fliessen.

\begin{table}[!h]
\caption{Stand der Wasseruhr}
\label{tab:standWasseruhr}
\centering
	\begin{tabular}{|l|l|l|}
		\hline
		Zähler & links [$m^3$] & rechts [$m^3$] \\
		\hline
		Vor spülen & 61.4277 & 86.2430 \\
		\hline
		Nach spülen & 61.4510 & 86.2655 \\
		\hline
	\end{tabular}
\end{table}

Danach spülen wir die Probenahmgefässe mit dem jeweiligen Probewasser. Dabei
haben wir auch den Hahn zur Probewasserentnahme gespühlt. Wir entnehmen je ca.
2 Liter der drei Probeflüssigkeiten. Diese sind Britawasser, Kalkexwasser und
Leitungswasser.

\subsection{Kochen des Probewassers}
Je 200ml von jedem Probewasser geben wir zusammen mit 2 Siedesteinchen in
Erlenmayerkolben mit Schliff. Diese stellten wir auf den Kocher und lassen sie
während einer Stunde kochen.

\begin{table}[!h]
\caption{Kochzeiten}
\label{tab:kochzeiten}
\centering
	\begin{tabular}{|l|l|l|}
		\hline
		Flüssigkeit & Startzeit & Endzeit \\
		\hline
		Kalkexwasser & 13:43 & 14:43 \\
		\hline
		Leitungswasser & 13:45 & 14:45 \\
		\hline
		Brittawasser & 13.45 & 14:45 \\
		\hline
	\end{tabular}
\end{table}

Nach dem Kochen stellen wir bei Leitungswasser und Kalkexwasser weisse
Ausfällungen fest, die sich im Erlenmeyerkolben absetzten. Quantitativ sind die
Ausfällungen sowohl beim Leitungswasser als auch beim Kalkexwasser gleich gross.
Beim Brittawasser können wir keine Ausfällungen feststellen.
Nach dem Kochen lassen wir die Flüssigkeiten etwas abkühlen, verschliessen
sie mit einem Stopfen und kühlen sie im Wasserbad vollständig ab. Danach filtern
wir die Flüssigkeiten.

\subsection{Hydrogenkarbonatbestimmung durch Neutralisation mit Salzsäure}
Mit einer Vollpipette entnahmen wir von jeder der drei Probeflüssigkeiten und
den drei gekochten Probeflüssigkeiten 100ml und geben diese in Erlenmeyerkolben.
Zwischen allen Entnahmen spühlten wir die Vollpipette mit deionisiertem Wasser, um eine Verfälschung der
Messergebnisse zu verhindern. Zu jeder Probe fügen wir 2 Tropfen
Eriochromschwarz als Mischindikator und einen Rührmagneten. Danach titrieren
wir Salzsäure bis zum Farbumschlag von blaugrün nach hellrosa. Die Salzsäure hat eine Konzentration
von 0.1mol/l. Bis zum Farbumschlag verwenden wir folgende Mengen:

\begin{table}[!h]
\caption{HCl-Lösung}
\label{tab:hcl}
\centering
	\begin{tabular}{|l|l|}
		\hline
		Flüssigkeit & Menge [ml] \\
		\hline
		Kalkexwasser & 6.738 \\
		\hline
		Kalkexwasser, gekocht & 0.932 \\
		\hline
		Leitungswasser &  6.674 \\
		\hline
		Leitungswasser, gekocht & 1.012 \\
		\hline
		Brittawasser & 2.346 \\
		\hline
		Brittawasser, gekocht & 1.25 \\
		\hline
	\end{tabular}
\end{table}

\subsection{Komplexometrische Härtebestimmung}
Von jeder der Probeflüssigkeiten entnehmen wir mit einer Vollpipette 50ml und
geben diese in ein Glas des Titrierautomaten. Jede der Proben versetzen wir mit
10ml Pufferlösung und 1ml Eriochromschwarz-Lösung. Das Glas befestigen wir am
Titrierautomaten und starten der Messung. Der Automat gibt der Flüssigkeit
EDTA-Lösung hinzu. Die Konzentration der EDTA-Lösung ist
$c(Na_2-EDTA\cdot2H_2O)=0.1mol/l$ Wir stellen folgenden Verbrauch fest:

\begin{table}[!h]
\caption{EDTA-Verbrauch}
\label{tab:edtaverbrauch}
\centering
	\begin{tabular}{|l|l|}
		\hline
		Flüssigkeit & Menge [ml] \\
		\hline
		Kalkexwasser & 1.775 \\
		\hline
		Kalkexwasser, gekocht & 0.324 \\
		\hline
		Leitungswasser & 1.774 \\
		\hline
		Leitungswasser, gekocht & 0.326 \\
		\hline
		Brittawasser & 0.705 \\
		\hline
		Brittawasser, gekocht & 0.387 \\
		\hline
	\end{tabular}
\end{table}

\subsection{Gesamthärtebestimmung mittels Schnelltest}
Wir tauchen den Schnelltest gemäss Anleitung auf der Verpackung in die
Testflüssigkeit ein, schütteln sie, warten eine Minute und lesen die Werte ab.
Dabei stellen wir folgende Werte fest:

\begin{table}[!h]
\caption{Schnelltest}
\label{tab:schnelltest}
\centering
	\begin{tabular}{|l|l|}
		\hline
		Flüssigkeit & Resultat [$^\circ$fH] \\
		\hline
		Kalkexwasser & $x>45$ \\
		\hline
		Kalkexwasser, gekocht & $9<x>18$ \\
		\hline
		Leitungswasser & $x>45$ \\
		\hline
		Leitungswasser, gekocht & $9<x>18$ \\
		\hline
		Brittawasser & $9<x>18$ \\
		\hline
		Brittawasser, gekocht & $9<x>18$ \\
		\hline
	\end{tabular}
\end{table}

\subsection{Beurteilung des Filterrückstandes}
Nach dem abkochen der Probeflüssigkeiten haben wir diese gefiltert. Bei
Kalkexwasser und Leitungswasser blieb Kalk im Filter zurück. Dieses lösen wir in
wenig deionisiertem Wasser und betrachten es unter dem Mikroskop.
Dabei können wir keine Unterschiede zwischen den beiden Rückständen feststellen.

\subsection{Interpretation}
Die Nichtkarbonathärte lässt sich auf zwei verschiedene Möglichkeiten bestimmen.
Durch den Kochvorgang wird die Karbonathärte ausgefällt. Die gekochten Proben
haben eine tiefe Karbonathärte, die Nichtkarbonathärte verändert sich aber
während dem Kochvorgang nicht. Die andere, hier verwendete Methode ist die
Berechnung mit der Formel:\\
Nichtkarbonathärte = Gesamthärte - Karbonathärte verwenden.\\
Bei unseren Messungen sind Abweichungen davon aufgetreten. Diese führen wir auf
Messungenauigkeiten zurück.

Die getätigten Messungen zeigen ganz klar dass die Wasserenthärtung mit
Britafilter funktioniert. Wir konnten keine Ausfällungen beim Abkochen des
Brittawassers Messen, zudem ist die Härte wesentlich tiefer als bei
Leitungswasser. Bei Kalkexwasser hingegen stellten wir genau die gleichen
Messwerte fest und beobachteten genau das gleiche wie beim Leitungswasser.
Daraus schliessen wir dass die Wasserenthärtung mit dem Kalkex-Prinzip nicht
funktioniert und daher keine Investition wert ist.


\subsection{Berechnungsrundlagen für Härtebestimmung}
Die Folgende Formel haben wir verwendet um die Resultate der Titration in
Wasserhärte umzu rechnen.
\begin{multicols}{3}
	$$V=\textrm{Volumen}$$
	$$c=\textrm{Konzentration}$$
	$$n=\textrm{Stoffmenge}$$
	$$HCO_3^-=\textrm{Hydrogenkarbonat}$$
	$$HCl=\textrm{Salzsäure}$$
\end{multicols}
	\begin{figure}[!h]
	\caption{Bestimmung der Hydogenkarbonatkonzentration}
	\label{equ:konezantration}
		$$c\left(HCO_3^-\right)=\frac{n\left(HCO_3^-\right)}{V\left(Probeloesung\right)}=
		\frac{V\left(HCl\right)\cdot c \left(HCl\right)}{V\left(Probeloesung\right)}$$
	\end{figure}
	
	\begin{figure}[!h]
	\caption{Bestimmung der Karbonathärte}
	\label{equ:karbonathaerte}
		$$c\left(KH\right)=\frac{1}{2}c\left(HCO_3^-\right)$$
	\end{figure}
	
	\begin{figure}[!h]
	\caption{Bestimmung der Gesamthärte}
	\label{equ:gesamthaerte}
		$$c\left(GH\right)=\frac{n\left(GH\right)}{V\left(\textrm{Probeloesung}\right)}=\frac{V\left(EDTA\right)\cdot
		c\left(EDTA\right)}{V\left(\textrm{Probeloesung}\right)}$$
	\end{figure}
	
	\begin{figure}[!h]
	\caption{Bestimmung der Nichtkarbonathärte}
	\label{equ:nichtkarbonat}
		$$c\left(NKH\right)=c\left(GH\right)-c\left(KH\right)$$
	\end{figure}
	\newpage
	
	\subsection{Resultate der Härtebestimmung}
	\begin{table}[!h]
	\caption{Resultate der Härtebestimmung in $\frac{mmol}{l}$}
	\label{tab:ResultateMmol}
	\begin{tabular}{|l|c|c|c|c|}
		\hline
			$\left[\frac{mmol}{l}\right]$ & Hydrogenkarbonat & Karbonathärte &
			Gesamthärte & Nichtkarbonathärte\\
		\hline
			Kalkexwasser & 6.738 & 3.369 & 3.55 & 0.181 \\
		\hline
			Kalkexwasser gekocht & 0.9320 & 0.466  & 0.648 & 0.182 \\
		\hline
			Leitungswasser & 6.674 & 3.337 & 3.548 & 0.211\\
		\hline
			Leitungswasser gekocht & 1.012 & 0.506 & 0.652 & 0.146\\
		\hline
			Brittawasser & 2.346 & 1.173 & 1.41 & 0.237 \\
		\hline
			Brittawasser gekocht & 1.25 & 0.625 & 0.774 & 0.149\\
		\hline
	\end{tabular}
	\end{table} 

		\begin{table}[!h]
	\caption{Resultate der Härtebestimmung in $fH$}
	\label{tab:ResultateFranz}
	\begin{tabular}{|l|c|c|c|}
		\hline
			$fH$ & Karbonathärte & Gesamthärte
			& Nichtkarbonathärte\\
		\hline
			Kalkexwasser & 33.69 & 35.5 & 1.81 \\
		\hline
			Kalkexwasser gekocht & 4.66 & 6.48 & 1.82 \\
		\hline
			Leitungswasser & 33.37 & 35.48 & 2.11\\
		\hline
			Leitungswasser gekocht & 5.06 & 6.52 & 1.46\\
		\hline
			Brittawasser & 11.73 & 14.1 & 2.37 \\
		\hline
			Brittawasser gekocht & 6.25 & 7.74 & 1.49\\
		\hline
	\end{tabular}
	\end{table} 

	\begin{figure}[!h]
	\caption{Matlab Code zur Berechnung der Wasserhärte}
	\label{code:matlabFile}
	\begin{lstlisting}
		close all;
		clear all;
		
		VolProbeHCL = 0.10; %l
		VolProbeKomplex = 0.05; %l
		
		%Kalkex: Kalkex gekocht; Leitung; Letiung gekocht; 
		%	Britta; Britta gekocht
		HCLVerbauch = [6.738 0.932 6.674 1.012 2.346 1.25]; 
		EDTAVerbauch = [ 1.775 0.324 1.774 0.326 0.705 0.387];
		
		StoffHCL = 0.1; %mol/l
		StoffEDTA = 0.1; %mol/l
		
		%Hydogenkarbonat konzentration
		for i=1:6
		    cHCO(i)=(StoffHCL*HCLVerbauch(i))/VolProbeHCL;
		end
		
		%Karbonathaerte
		for i=1:6
		    cKH(i)= 0.5*(cHCO(i));
		end
		
		%Gesamthaerte
		for i=1:6
		    cGH(i)=(EDTAVerbauch(i)*StoffEDTA)/VolProbeKomplex;
		end
		
		%Nichtkarbonathaerte
		for i=1:6
		    cNKH(i)=cGH(i)-cKH(i);
		end
		
		disp('Kalkex, Kalkex gekocht; Leitung; 
			Leitung gekocht; Britta; Britta gekocht')
		cHCO
		cKH
		cGH
		cNKH
	\end{lstlisting}
	\end{figure}




\end{document}
